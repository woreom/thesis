\documentclass[12pt,a4paper,BCOR=.7cm,headsepline,bibliography=totoc]{report}

\title{تحقیق در ویرایش ژنوم به کمک تناوب‌هایِ کوتاهِ پالیندرومِ فاصله‌دارِ منظمِ خوشه‌ای}

\usepackage{PersianTemplate}

\newcommand{\specialcell}[2][c]{%
  \begin{tabular}[#1]{@{}c@{}}#2\end{tabular}}

\makeindex

\logo{pictures/sharif-logo.png}
\university{دانشگاه صنعتی شریف }
\department{دانشکده علوم ریاضی}
\thesis{ پایان نامه کارشناسی‌ارشد}
\subject{‌ریاضی کاربردی}

\author{محمد رستمی }
\supervisor{دکتر محسن شریفی تبار}
\secsupervisor{دکتر حمیدرضا ربیعی}%در صورت نیاز
\advisor{دکتر محمدحسین رهبان}%در صورت نیاز
\date{\today}


\begin{document}
\makethesistitle
\begin{abstract}
تناوب‌هایِ کوتاهِ پالیندرومِ فاصله‌دارِ منظمِ خوشه‌ای یا به طور خلاصه ، \lr{CRISPR} (کریسپر) یکی از روش‌های نسبتا نوین است که متخصصان ژنتیک و محققان پزشکی را قادر می سازد تا با حذف بخشهایی از ژنوم ،
افزودن یا تغییر بخش هایی از آن در \lr{dna} (دی‌ان‌ای) تغییر ایجاد کنند. این فناوری نوعی سیستم ایمنی تطابق‌پذیر در باکتری‌ها است که با کمک آن می‌توان بسیاری از بیماری‌ها مانند نابینوایی و ناشنوایی و حتی سرطان را درمان کرد. یکی از مشکلات بزرگ در استفاده موفق کریسپر، دقیق پیش‌بینی کردن تاثیر \lr{Guide RNA} (راهنمای آر‌ان‌ای) روی هدف و حساسیت خارج از هدف است. در حالی که برخی از روش ها این طرح ها را طبقه بندی می کنند ، بیشتر
الگوریتم ها بر روی داده های جداگانه با ژن ها و سلول های مختلف هستند. عدم تعمیم این روش ها
مانع استفاده از این راهنما در آزمایشات بالینی می شود ، زیرا برای هر درمان، این فرایند باید دقیقا برای همان سلول درست شده باشد و عموما داده کافی برای طراحی الگوریتم در آن سلول در دسترس نیست. ما سعی می‌کنیم مشکل تعمیم‌پذیری را حل کنیم و مدل‌ای ارائه دهیم که هم به‌صورت عمومی و هدفمند دقت خوبی داشته باشد تا که محققان در بهینه سازی طراح راهنمای آران‌ای با حساسیت مناسب کمک کند.
\end{abstract}

\end{document}